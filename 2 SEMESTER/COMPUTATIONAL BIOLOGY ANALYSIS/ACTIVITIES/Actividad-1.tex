% Options for packages loaded elsewhere
\PassOptionsToPackage{unicode}{hyperref}
\PassOptionsToPackage{hyphens}{url}
%
\documentclass[
]{article}
\usepackage{amsmath,amssymb}
\usepackage{lmodern}
\usepackage{iftex}
\ifPDFTeX
  \usepackage[T1]{fontenc}
  \usepackage[utf8]{inputenc}
  \usepackage{textcomp} % provide euro and other symbols
\else % if luatex or xetex
  \usepackage{unicode-math}
  \defaultfontfeatures{Scale=MatchLowercase}
  \defaultfontfeatures[\rmfamily]{Ligatures=TeX,Scale=1}
\fi
% Use upquote if available, for straight quotes in verbatim environments
\IfFileExists{upquote.sty}{\usepackage{upquote}}{}
\IfFileExists{microtype.sty}{% use microtype if available
  \usepackage[]{microtype}
  \UseMicrotypeSet[protrusion]{basicmath} % disable protrusion for tt fonts
}{}
\makeatletter
\@ifundefined{KOMAClassName}{% if non-KOMA class
  \IfFileExists{parskip.sty}{%
    \usepackage{parskip}
  }{% else
    \setlength{\parindent}{0pt}
    \setlength{\parskip}{6pt plus 2pt minus 1pt}}
}{% if KOMA class
  \KOMAoptions{parskip=half}}
\makeatother
\usepackage{xcolor}
\usepackage[margin=1in]{geometry}
\usepackage{color}
\usepackage{fancyvrb}
\newcommand{\VerbBar}{|}
\newcommand{\VERB}{\Verb[commandchars=\\\{\}]}
\DefineVerbatimEnvironment{Highlighting}{Verbatim}{commandchars=\\\{\}}
% Add ',fontsize=\small' for more characters per line
\usepackage{framed}
\definecolor{shadecolor}{RGB}{248,248,248}
\newenvironment{Shaded}{\begin{snugshade}}{\end{snugshade}}
\newcommand{\AlertTok}[1]{\textcolor[rgb]{0.94,0.16,0.16}{#1}}
\newcommand{\AnnotationTok}[1]{\textcolor[rgb]{0.56,0.35,0.01}{\textbf{\textit{#1}}}}
\newcommand{\AttributeTok}[1]{\textcolor[rgb]{0.77,0.63,0.00}{#1}}
\newcommand{\BaseNTok}[1]{\textcolor[rgb]{0.00,0.00,0.81}{#1}}
\newcommand{\BuiltInTok}[1]{#1}
\newcommand{\CharTok}[1]{\textcolor[rgb]{0.31,0.60,0.02}{#1}}
\newcommand{\CommentTok}[1]{\textcolor[rgb]{0.56,0.35,0.01}{\textit{#1}}}
\newcommand{\CommentVarTok}[1]{\textcolor[rgb]{0.56,0.35,0.01}{\textbf{\textit{#1}}}}
\newcommand{\ConstantTok}[1]{\textcolor[rgb]{0.00,0.00,0.00}{#1}}
\newcommand{\ControlFlowTok}[1]{\textcolor[rgb]{0.13,0.29,0.53}{\textbf{#1}}}
\newcommand{\DataTypeTok}[1]{\textcolor[rgb]{0.13,0.29,0.53}{#1}}
\newcommand{\DecValTok}[1]{\textcolor[rgb]{0.00,0.00,0.81}{#1}}
\newcommand{\DocumentationTok}[1]{\textcolor[rgb]{0.56,0.35,0.01}{\textbf{\textit{#1}}}}
\newcommand{\ErrorTok}[1]{\textcolor[rgb]{0.64,0.00,0.00}{\textbf{#1}}}
\newcommand{\ExtensionTok}[1]{#1}
\newcommand{\FloatTok}[1]{\textcolor[rgb]{0.00,0.00,0.81}{#1}}
\newcommand{\FunctionTok}[1]{\textcolor[rgb]{0.00,0.00,0.00}{#1}}
\newcommand{\ImportTok}[1]{#1}
\newcommand{\InformationTok}[1]{\textcolor[rgb]{0.56,0.35,0.01}{\textbf{\textit{#1}}}}
\newcommand{\KeywordTok}[1]{\textcolor[rgb]{0.13,0.29,0.53}{\textbf{#1}}}
\newcommand{\NormalTok}[1]{#1}
\newcommand{\OperatorTok}[1]{\textcolor[rgb]{0.81,0.36,0.00}{\textbf{#1}}}
\newcommand{\OtherTok}[1]{\textcolor[rgb]{0.56,0.35,0.01}{#1}}
\newcommand{\PreprocessorTok}[1]{\textcolor[rgb]{0.56,0.35,0.01}{\textit{#1}}}
\newcommand{\RegionMarkerTok}[1]{#1}
\newcommand{\SpecialCharTok}[1]{\textcolor[rgb]{0.00,0.00,0.00}{#1}}
\newcommand{\SpecialStringTok}[1]{\textcolor[rgb]{0.31,0.60,0.02}{#1}}
\newcommand{\StringTok}[1]{\textcolor[rgb]{0.31,0.60,0.02}{#1}}
\newcommand{\VariableTok}[1]{\textcolor[rgb]{0.00,0.00,0.00}{#1}}
\newcommand{\VerbatimStringTok}[1]{\textcolor[rgb]{0.31,0.60,0.02}{#1}}
\newcommand{\WarningTok}[1]{\textcolor[rgb]{0.56,0.35,0.01}{\textbf{\textit{#1}}}}
\usepackage{graphicx}
\makeatletter
\def\maxwidth{\ifdim\Gin@nat@width>\linewidth\linewidth\else\Gin@nat@width\fi}
\def\maxheight{\ifdim\Gin@nat@height>\textheight\textheight\else\Gin@nat@height\fi}
\makeatother
% Scale images if necessary, so that they will not overflow the page
% margins by default, and it is still possible to overwrite the defaults
% using explicit options in \includegraphics[width, height, ...]{}
\setkeys{Gin}{width=\maxwidth,height=\maxheight,keepaspectratio}
% Set default figure placement to htbp
\makeatletter
\def\fps@figure{htbp}
\makeatother
\setlength{\emergencystretch}{3em} % prevent overfull lines
\providecommand{\tightlist}{%
  \setlength{\itemsep}{0pt}\setlength{\parskip}{0pt}}
\setcounter{secnumdepth}{-\maxdimen} % remove section numbering
\ifLuaTeX
  \usepackage{selnolig}  % disable illegal ligatures
\fi
\IfFileExists{bookmark.sty}{\usepackage{bookmark}}{\usepackage{hyperref}}
\IfFileExists{xurl.sty}{\usepackage{xurl}}{} % add URL line breaks if available
\urlstyle{same} % disable monospaced font for URLs
\hypersetup{
  pdftitle={Actividad en clase \textbar{} Ejercicios básicos de R},
  pdfauthor={Diego Espejo \& Daniel Esparza},
  hidelinks,
  pdfcreator={LaTeX via pandoc}}

\title{Actividad en clase \textbar{} Ejercicios básicos de R}
\author{Diego Espejo \& Daniel Esparza}
\date{2023-03-31}

\begin{document}
\maketitle

\hypertarget{ejercicios-buxe1sicos-de-r}{%
\section{Ejercicios Básicos de R}\label{ejercicios-buxe1sicos-de-r}}

\hypertarget{instrucciones}{%
\subsubsection{Instrucciones}\label{instrucciones}}

\begin{enumerate}
\def\labelenumi{\arabic{enumi}.}
\item
  Crear un vector x con los datos 10, 11, 13, -1, 6, 3

\begin{Shaded}
\begin{Highlighting}[]
\NormalTok{x }\OtherTok{\textless{}{-}} \FunctionTok{c}\NormalTok{(}\DecValTok{10}\NormalTok{, }\DecValTok{11}\NormalTok{, }\DecValTok{13}\NormalTok{, }\SpecialCharTok{{-}}\DecValTok{1}\NormalTok{, }\DecValTok{6}\NormalTok{, }\DecValTok{3}\NormalTok{)}
\NormalTok{x}
\end{Highlighting}
\end{Shaded}

\begin{verbatim}
## [1] 10 11 13 -1  6  3
\end{verbatim}
\item
  Calcula estadísticas simples de x. Calcular la media, la desviación
  estándar y la varianza. Crear un objeto (vector) con el nombre est.x
  en el que guardar los 3 estadísticos.

\begin{Shaded}
\begin{Highlighting}[]
\NormalTok{media }\OtherTok{\textless{}{-}} \FunctionTok{mean}\NormalTok{(x)}
\NormalTok{media}
\end{Highlighting}
\end{Shaded}

\begin{verbatim}
## [1] 7
\end{verbatim}

\begin{Shaded}
\begin{Highlighting}[]
\NormalTok{sd }\OtherTok{\textless{}{-}} \FunctionTok{sd}\NormalTok{(x)}
\NormalTok{sd}
\end{Highlighting}
\end{Shaded}

\begin{verbatim}
## [1] 5.329165
\end{verbatim}

\begin{Shaded}
\begin{Highlighting}[]
\NormalTok{var }\OtherTok{\textless{}{-}} \FunctionTok{var}\NormalTok{(x)}
\NormalTok{var}
\end{Highlighting}
\end{Shaded}

\begin{verbatim}
## [1] 28.4
\end{verbatim}

\begin{Shaded}
\begin{Highlighting}[]
\NormalTok{est.x }\OtherTok{\textless{}{-}} \FunctionTok{c}\NormalTok{(media, sd, var)}
\end{Highlighting}
\end{Shaded}
\item
  Escribe un programa R para crear una secuencia de números del 20 al 50
  y encuentre la media de los números del 20 al 60 y la suma de los
  números del 51 al 91. Tip: utiliza las funciones de R.
\end{enumerate}

\begin{verbatim}
```r
20:50
```

```
##  [1] 20 21 22 23 24 25 26 27 28 29 30 31 32 33 34 35 36 37 38 39 40 41 42 43 44
## [26] 45 46 47 48 49 50
```

```r
mean(20:60)
```

```
## [1] 40
```

```r
sum(51:91)
```

```
## [1] 2911
```
\end{verbatim}

\begin{enumerate}
\def\labelenumi{\arabic{enumi}.}
\setcounter{enumi}{3}
\tightlist
\item
  Escribe un programa R para crear un vector que contenga 10 valores
  enteros aleatorios entre -100 y +50. Revisa la función sample
  (\url{https://www.rdocumentation.org/packages/base/versions/3.6.2/topics/sample})
\end{enumerate}

\begin{verbatim}
```r
rand <- sample(-100:50, 10)
rand
```

```
##  [1] -77  -6  24  -8  31 -31 -38 -47 -57 -89
```
\end{verbatim}

\begin{enumerate}
\def\labelenumi{\arabic{enumi}.}
\setcounter{enumi}{4}
\tightlist
\item
  Escribe un programa R para obtener los primeros 10 números de
  Fibonacci (\url{https://www.ecured.cu/N\%C3\%BAmeros_de_Fibonacci}).
\end{enumerate}

\begin{verbatim}
Como código base tienes:

fufb \<- numeric(10)

fb[1] \<- fb[2] \<- 1

Practica usando un for para terminar tu código.


```r
fb <- numeric(10)
fb[1] <- fb[2] <- 1

count <- 1
for (value in fb) {
  if (value != 1) {
    fb[count] <- fb[count-1] + fb[count - 2]
  }
  count <- count + 1
}
fb
```

```
##  [1]  1  1  2  3  5  8 13 21 34 55
```
\end{verbatim}

\begin{enumerate}
\def\labelenumi{\arabic{enumi}.}
\setcounter{enumi}{5}
\tightlist
\item
  Escribe un programa R para encontrar el valor máximo y mínimo de un
  vector dado. Debes probar con:
\end{enumerate}

\begin{verbatim}
a.  c(10, 20, 30, 4, 50, -60)

b.  c(10, 20, 30, 4, 50, -60)


```r
a <- c(10, 20, 30, 4, 50, -60)
b <- c(10, 20, 30, 4, 50, 100) 

max(a)
```

```
## [1] 50
```

```r
min(a)
```

```
## [1] -60
```

```r
max(b)
```

```
## [1] 100
```

```r
min(b)
```

```
## [1] 4
```
\end{verbatim}

\begin{enumerate}
\def\labelenumi{\arabic{enumi}.}
\setcounter{enumi}{6}
\item
  Escribe una función R para multiplicar dos vectores de tipo entero y
  longitud n, de la misma longitud ambos.

  \begin{enumerate}
  \def\labelenumii{\alph{enumii}.}
  \tightlist
  \item
    multiplica(c(10, 20), c(3,4)) \# salida: {[}1{]} 30 80
  \end{enumerate}

\begin{Shaded}
\begin{Highlighting}[]
\NormalTok{v1 }\OtherTok{\textless{}{-}} \FunctionTok{c}\NormalTok{(}\DecValTok{10}\NormalTok{, }\DecValTok{20}\NormalTok{)}
\NormalTok{v2 }\OtherTok{\textless{}{-}} \FunctionTok{c}\NormalTok{(}\DecValTok{3}\NormalTok{, }\DecValTok{4}\NormalTok{)}

\NormalTok{v1 }\SpecialCharTok{*}\NormalTok{ v2}
\end{Highlighting}
\end{Shaded}

\begin{verbatim}
## [1] 30 80
\end{verbatim}
\item
  Escribe una función R para contar el valor específico en un vector
  dado.
\end{enumerate}

\begin{verbatim}
a.  cuenta(c(10, 20, 10, 7, 24,7, 5),7) \# salid  a: [1] 2


```r
vector <- c(10, 20, 10, 7, 24,7, 5)

cuenta <- function(vector, num) {
  count <- 0
  for (element in vector) {
    if (element == num) {
      count <- count + 1
    }
  }
  return(count)
}

cuenta(vector, 7)
```

```
## [1] 2
```
\end{verbatim}

\begin{enumerate}
\def\labelenumi{\arabic{enumi}.}
\setcounter{enumi}{8}
\item
  Escribe una función en R para extraer cada enésimo elemento de un
  vector dado. Un prueba es:

  \begin{enumerate}
  \def\labelenumii{\alph{enumii}.}
  \item
    v \textless- 1:100
  \item
    enesimo(v, 5)
  \item
    \# Salida: {[}1{]}~ 1~6 11 16 21 26 31 36 41 46 51 56 61 66 71 76 81
    86 91 96
  \end{enumerate}

\begin{Shaded}
\begin{Highlighting}[]
\NormalTok{v }\OtherTok{\textless{}{-}} \DecValTok{1}\SpecialCharTok{:}\DecValTok{100}

\NormalTok{enesimo }\OtherTok{\textless{}{-}} \ControlFlowTok{function}\NormalTok{(v, paso) \{}
\NormalTok{  elementos }\OtherTok{\textless{}{-}} \FunctionTok{seq}\NormalTok{(}\AttributeTok{from =} \DecValTok{1}\NormalTok{, }\AttributeTok{to =} \FunctionTok{length}\NormalTok{(v), }\AttributeTok{by =}\NormalTok{ paso)}
  \FunctionTok{return}\NormalTok{(v[elementos])}
\NormalTok{\}}
\FunctionTok{enesimo}\NormalTok{(v, }\DecValTok{5}\NormalTok{)}
\end{Highlighting}
\end{Shaded}

\begin{verbatim}
##  [1]  1  6 11 16 21 26 31 36 41 46 51 56 61 66 71 76 81 86 91 96
\end{verbatim}
\end{enumerate}

\end{document}
